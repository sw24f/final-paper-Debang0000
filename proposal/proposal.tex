\documentclass[12pt]{article}

\usepackage{amsmath}
\usepackage[margin=1in]{geometry}
\usepackage{graphicx}
\usepackage{booktabs}
\usepackage{natbib}

\usepackage[colorlinks=true, citecolor=blue]{hyperref}

\title{Proposal: Predicting Recidivism and Improving Rehabilitation}
\author{Debang Ou\\
  University of Connecticut
}
\begin{document}
\maketitle
\paragraph{Introduction}
Recidivism, or the act of reoffending after release from incarceration, remains one of the most significant issues in the criminal justice system. High recidivism rates in the U.S. complicate rehabilitation efforts and pose challenges for public safety. While probation and parole are key tools in mitigating reoffending risks, they often fail to prevent violations that result in recidivism \citep{kaeble2020probation}.

This paper aims to explore three questions (may change in the future):
\begin{enumerate}
    \item What is the role of probation and parole violations in recidivism, and how can we better forecast these violations?
    \item What is the role of drug use and gang affiliation in recidivism, and how can we prevent such recidivism before it happens.
    \item How can prediction models achieve a balance between accuracy and fairness \citep{berk2020acceptable}?
\end{enumerate}


\paragraph{Specific Aims}
This study will focus on the following key objectives:
\begin{enumerate}
    \item Investigate the contribution of probation and parole violations to recidivism, identifying predictive patterns that may help prevent these violations \citep{kaeble2020probation}.
    \item Investigate the role of drug use and gang affiliation in contributing to recidivism rates, focusing on analyzing how these factors interact with other behavioral and demographic variables to increase the risk of recidivism, and how predictive models can be developed to support preventative measures before reoffending occurs.
    \item Evaluate and improve the fairness and accuracy of existing criminal justice risk assessment models, focusing on finding balance between increasing accuracy and minimizing biases \citep{berk2020acceptable}.
\end{enumerate}

\paragraph{Data}
Data for this research will be drawn from the Bureau of Justice Statistics.\\
NIJ's Recidivism Challenge Full Dataset\\
Data Provided by Georgia Department of Community Supervision, \\Georgia Crime Information Center\\
Last Updated: July 15, 2021

\paragraph{Research Design and Methods}
This part is yet undecided. Will determine it after the preliminary analysis on the data is done. Techniques like LASSO and cross-validation will be used. 

\paragraph{Discussion}
The main challenge in this research is to ensure that the prediction model remains both accurate and fair. There is growing concern that such models may reinforce biases, particularly related to race and socioeconomic status \citep{berk2020acceptable}.

\bibliography{refs}
\bibliographystyle{chicago}

\end{document}